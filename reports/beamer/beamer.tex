% Classe du document
\documentclass[aspectratio=169]{beamer}  % Format 16:9 pour un affichage moderne

% Thème et couleurs
\usetheme{Madrid}
\usecolortheme{whale}
\useinnertheme{rectangles}
\useoutertheme{miniframes}
\setbeamertemplate{navigation symbols}{}  % Supprime les symboles de navigation

% Packages essentiels
\usepackage[utf8]{inputenc}
\usepackage[T1]{fontenc}
\usepackage[french]{babel}
\usepackage{lmodern}
\usepackage{amsmath,amssymb,amsthm}
\usepackage{graphicx}
\usepackage{pgfplots}
\usepackage{listings}  % Pour le code
\pgfplotsset{compat=1.18}

% Informations du document
\title[Impact des nouvelles sur les LOB]{Impact des nouvelles sur les \textit{Limit Order Books}}
\subtitle{Une analyse par le modèle Queue-Reactive}
\author[LAFERTE \& AIAD]{LAFERTE Edouard \and AIAD Janis}
\institute[École Polytechnique]{
    Département de mathématiques appliquées\\
    École Polytechnique
}
\date{Juin 2024}

\begin{document}

% Page de titre
\begin{frame}
    \titlepage
\end{frame}

\begin{frame}{Les marchés haute fréquence}
    \begin{itemize}
        \item Électronification des marchés financiers
        \item Réduction drastique du temps d'exécution
        \item Trois catégories de trading:
        \begin{itemize}
            \item \textbf{Low Frequency (LF)}: Échelle de plusieurs mois
            \item \textbf{Mid Frequency (MF)}: Échelle journalière
            \item \textbf{High Frequency (HF)}: Échelle microseconde
        \end{itemize}
    \end{itemize}
\end{frame}

\begin{frame}{Études empiriques}
    \begin{columns}
        \begin{column}{0.6\textwidth}
            \textbf{Impact des annonces sur le marché}
            \begin{itemize}
                \item \textbf{Volatilité}:
                \begin{itemize}
                    \item Augmentation de 150-300\%
                    \item Pics lors des annonces majeures
                \end{itemize}
                \item \textbf{Liquidité}:
                \begin{itemize}
                    \item Modification des spreads bid-ask
                    \item Changements de profondeur du LOB
                \end{itemize}
            \end{itemize}
        \end{column}
        \begin{column}{0.4\textwidth}
            \begin{alertblock}{Source}
                \small{Review of Financial Studies (2019)}
                \begin{itemize}
                    \item Étude sur impact des news
                    \item Analyse haute fréquence
                    \item Données multi-marchés
                \end{itemize}
            \end{alertblock}
        \end{column}
    \end{columns}
\end{frame}

\begin{frame}{Comportement des Traders et Microstructure}
    \begin{columns}
        \begin{column}{0.6\textwidth}
            \textbf{Déroulé d'une annonce}
            \begin{itemize}
                \item \textbf{Pré-annonce}:
                \begin{itemize}
                    \item Ordres limites prudents
                    \item Positionnement stratégique
                    \item Réduction de l'exposition
                \end{itemize}
                \item \textbf{Post-annonce}:
                \begin{itemize}
                    \item Ordres marchés agressifs
                    \item Exploitation de l'information
                    \item Augmentation de la volatilité
                \end{itemize}
            \end{itemize}
        \end{column}
        \begin{column}{0.4\textwidth}
            \begin{alertblock}{Impact sur le LOB}
                \begin{itemize}
                    \item Volatilité +150-300\%
                    \item Modification des spreads
                    \item Changement de profondeur
                \end{itemize}
            \end{alertblock}
        \end{column}
    \end{columns}
\end{frame}

\begin{frame}{Imbalance et Prédiction}
    \begin{columns}
        \begin{column}{0.6\textwidth}
            \textbf{Mesure d'imbalance}
            \begin{equation*}
                \text{Imb}_t = \frac{Q^{best\ bid}_t-Q^{best\ ask}_t}{Q^{best\ bid}_t+Q^{best\ ask}_t}
            \end{equation*}
            \begin{itemize}
                \item \textbf{Pouvoir prédictif}:
                \begin{itemize}
                    \item Indicateur avancé des prix
                    \item Forte corrélation avec mouvements
                    \item Efficace sur spread 1 tick
                \end{itemize}
            \end{itemize}
        \end{column}
        \begin{column}{0.4\textwidth}
            \begin{alertblock}{Choix des actifs}
                \begin{itemize}
                    \item Focus spread 1 tick
                    \item Réduction du bruit
                    \item Meilleure prédictibilité
                \end{itemize}
            \end{alertblock}
        \end{column}
    \end{columns}
\end{frame}

\begin{frame}{Problématique}
    \begin{itemize}
        \item Détection d'événements rares sur les marchés financiers
        \item Modélisation des mouvements de prix
        \item Prédiction des sauts de prix
    \end{itemize}
\end{frame}

% Plan
\begin{frame}{Plan de la présentation}
    \tableofcontents
\end{frame}

% Section 0: Données et Méthodologie
\section{Données et Méthodologie}

\begin{frame}{Sources des données}
    \begin{itemize}
        \item \textbf{Nasdaq Databento}:
        \begin{itemize}
            \item Données haute fréquence (microsecondes)
            \item Format parquet optimisé
            \item Messages d'ordres complets (ITCH)
        \end{itemize}
        \item \textbf{Chicago Mercantile Exchange}:
        \begin{itemize}
            \item Données futures et options
            \item Granularité temporelle fine
        \end{itemize}
    \end{itemize}
\end{frame}

\begin{frame}{Cas particulier de RIOT}
    \begin{columns}
        \begin{column}{0.6\textwidth}
            \textbf{Deux régimes distincts observés}:
            \begin{itemize}
                \item \textbf{Régime 1}: Trading classique
                \begin{itemize}
                    \item Dynamique de marché standard
                    \item Variation des prix normale
                \end{itemize}
                \item \textbf{Régime 2}: Trading suspect
                \begin{itemize}
                    \item Prix constants sur longues périodes
                    \item Volumes d'échange importants
                    \item Potentiels accords hors marché
                \end{itemize}
            \end{itemize}
        \end{column}
        \begin{column}{0.4\textwidth}
            \begin{alertblock}{Particularités CME}
                \begin{itemize}
                    \item Transactions répétées au même prix
                    \item Volumes prédéfinis
                    \item Possible entente entre acteurs
                    \item Marché Chicago spécifique
                \end{itemize}
            \end{alertblock}
        \end{column}
    \end{columns}
    \vspace{0.2cm}
    \begin{itemize}
        \item \textbf{Implications}:
        \begin{itemize}
            \item Nécessité de filtrer ces périodes
            \item Biais potentiel dans l'analyse
            \item Question de la manipulation de marché
        \end{itemize}
    \end{itemize}
\end{frame}

\begin{frame}{Prétraitement systématique des données}
    \begin{itemize}
        \item \textbf{Pipeline systématique de nettoyage}:
        \begin{itemize}
            \item Filtrage des données aberrantes
            \item Synchronisation temporelle
            \item Reconstruction du carnet d'ordres
        \end{itemize}
        \item \textbf{Hypothèses validées}:
        \begin{itemize}
            \item Indépendance sur périodes sans drift
            \item Validation empirique sur données
            \item Stationnarité des intensités
        \end{itemize}
    \end{itemize}
\end{frame}

\begin{frame}{Infrastructure technique}
    \begin{columns}
        \begin{column}{0.6\textwidth}
            \textbf{Traitement des données (100 Go)}:
            \begin{itemize}
                \item \textbf{Librairies haute performance}:
                \begin{itemize}
                    \item cuDF: Traitement GPU natif
                    \item Polars: Optimisation CPU
                    \item Polars-CUDA: Accélération GPU
                \end{itemize}
                \item \textbf{Environnement de calcul}:
                \begin{itemize}
                    \item Jupyter notebooks interactifs
                    \item Calculs distribués sur cluster
                    \item Optimisation mémoire et CPU/GPU
                \end{itemize}
            \end{itemize}
        \end{column}
        \begin{column}{0.4\textwidth}
            \begin{alertblock}{Librairie open-source}
                \begin{itemize}
                    \item github.com/janisaiad/HFT\_QR\_RL
                    \item Implémentation QR
                    \item Outils d'analyse
                    \item Pipeline de simulation
                \end{itemize}
            \end{alertblock}
        \end{column}
    \end{columns}
    \vspace{0.2cm}
    \begin{itemize}
        \item \textbf{Avantages}:
        \begin{itemize}
            \item Traitement parallèle efficace
            \item Gestion optimisée de la mémoire
            \item Réduction temps de calcul ×10-100
        \end{itemize}
    \end{itemize}
\end{frame}

\begin{frame}{Caractéristiques des données}
    \begin{table}
    \centering
    \begin{tabular}{|c|c|c|c|c|}
    \hline
    \textbf{Actif} & \textbf{Nb/Jour Actions} & \textbf{Pct Add} & \textbf{Pct Cancel} & \textbf{Pct Order} \\ \hline
    GOOGL & 2.67M & 49.65\% & 46.64\% & 3.70\% \\ \hline
    LCID & 70.7k & 45.69\% & 42.94\% & 11.35\% \\ \hline
    KHC & 203k & 47.75\% & 45.34\% & 6.89\% \\ \hline
    \end{tabular}
    \caption{Statistiques moyennes sur 65 jours (28 juillet - 4 novembre)}
    \end{table}
    \vspace{0.3cm}
    \begin{itemize}
        \item Données haute fréquence (microsecondes)
        \item Format parquet optimisé
        \item Messages d'ordres complets (ITCH)
        \item Période d'étude : 65 jours
    \end{itemize}
\end{frame}

\begin{frame}{Critères de filtrage des données}
    \begin{table}
    \centering
    \begin{tabular}{|c|c|c|}
    \hline
    \textbf{Critère} & \textbf{Seuil} & \textbf{Impact} \\ \hline
    Outliers & $\pm3\sigma$ des prix & Suppression points aberrants \\ \hline
    Missing data & $>5\%$ par jour & Interpolation linéaire \\ \hline
    Valeurs aberrantes & 614 dans les tailles & Remplacement par 0 \\ \hline
    Profondeur carnet & Niveau 0 & Analyse du BBO \\ \hline
    Publisher ID & 39 & Sélection publisher\_id 39 \\ \hline
    Horodatage & ts\_event & 14h-20h uniquement \\ \hline
    \end{tabular}
    \end{table}
    \vspace{0.3cm}
    \begin{itemize}
        \item Focus sur deux titres complémentaires:
        \begin{itemize}
            \item \textbf{GOOGL}: Volume élevé (2.67M/jour), tick fin
            \item \textbf{KHC}: Moins de trades (203k/jour), tick plus large
        \end{itemize}
        \item Validation systématique des modifications
        \item Contrôle de cohérence des actions
    \end{itemize}
\end{frame}

\begin{frame}{Pipeline de nettoyage}
    \begin{columns}
        \begin{column}{0.48\textwidth}
            \textbf{Étapes de validation}:
            \begin{itemize}
                \item Transactions (T):
                \begin{itemize}
                    \item Vérification volume échangé
                    \item Cohérence des diminutions
                \end{itemize}
                \item Ajouts (A):
                \begin{itemize}
                    \item Validation taille ajoutée
                    \item Correspondance exacte
                \end{itemize}
            \end{itemize}
        \end{column}
        \begin{column}{0.48\textwidth}
            \textbf{Traitement technique}:
            \begin{itemize}
                \item Cast en Int64
                \item Vérification colonnes
                \item Indicateurs qualité:
                \begin{itemize}
                    \item status\_N
                    \item status\_diff
                \end{itemize}
            \end{itemize}
        \end{column}
    \end{columns}
\end{frame}

% Section 1: Introduction aux Limit Order Books
\section{Introduction aux Limit Order Books}

\begin{frame}{Structure du Limit Order Book}
    \begin{columns}
        \begin{column}{0.6\textwidth}
            \begin{itemize}
                \item Trois types d'ordres:
                \begin{itemize}
                    \item \textbf{Limit Order}: Proposition à prix fixé
                    \item \textbf{Market Order}: Exécution immédiate
                    \item \textbf{Cancellation}: Retrait d'ordre
                \end{itemize}
                \item Organisation en files d'attente
                \item Prix de référence $p_{ref}$
            \end{itemize}
        \end{column}
        \begin{column}{0.4\textwidth}
            % Insérer ici un schéma simplifié du LOB
        \end{column}
    \end{columns}
\end{frame}

\begin{frame}{Point anecdotique sur l'asymétrie du LOB Bitcoin}
    \begin{figure}
        \centering
        \includegraphics[width=0.25\textwidth]{bitcoin.png}
        \caption{Structure asymétrique du Limit Order Book du Bitcoin}
    \end{figure}
\end{frame}

\begin{frame}{Explication de l'asymétrie Bitcoin}
    \begin{columns}
        \begin{column}{0.5\textwidth}
            \textbf{Côté Ask (Vente)}
            \begin{itemize}
                \item \small{Courbe concave}
                \begin{itemize}
                    \item \footnotesize{Accumulation d'ordres à prix élevés}
                    \item \footnotesize{Anticipation de hausses futures}
                    \item \footnotesize{Comportement HODL}
                \end{itemize}
                \item \small{Grosses queues en volume}
                \begin{itemize}
                    \item \footnotesize{Mineurs et early adopters}
                    \item \footnotesize{Positions historiques}
                \end{itemize}
            \end{itemize}
        \end{column}
        \begin{column}{0.5\textwidth}
            \textbf{Côté Bid (Achat)}
            \begin{itemize}
                \item \small{Courbe convexe}
                \begin{itemize}
                    \item \footnotesize{Concentration près du prix courant}
                    \item \footnotesize{Stratégie d'accumulation}
                    \item \footnotesize{Liquidité immédiate}
                \end{itemize}
                \item \small{Implications}
                \begin{itemize}
                    \item \footnotesize{Résistance aux mouvements haussiers}
                    \item \footnotesize{Volatilité asymétrique}
                \end{itemize}
            \end{itemize}
            \begin{alertblock}{Particularité}
                Cette structure unique reflète la nature spéculative et la psychologie particulière du marché crypto
            \end{alertblock}
        \end{column}
    \end{columns}
\end{frame}

% Section 2: Le modèle Queue-Reactive
\section{Le modèle Queue-Reactive}

\begin{frame}{Principes du modèle QR}
    \begin{itemize}
        \item Modélisation par files d'attente
        \item Prix fixe de référence $p_{ref}$
        \item Évolution selon processus de Markov
        \item Variables de Poisson indépendantes
    \end{itemize}
\end{frame}

\begin{frame}{Dynamique du modèle QR}
    \begin{columns}
        \begin{column}{0.6\textwidth}
            \begin{equation*}
                Q(t) = (Q_{-K}(t),...Q_{-1}(t),Q_{1}(t),...Q_K(t))
            \end{equation*}
            \begin{itemize}
                \item Processus de Poisson d'intensités $\lambda_i^U(Q_i(t))$
                \item $U \in \{\text{Add}, \text{Cancel}, \text{Trade}\}$
                \item \textbf{Mean-reversion double}:
                \begin{itemize}
                    \item $\theta$: retour au prix de référence
                    \item $\theta_{\text{reinit}}$: nouveau $p_{\text{ref}}$ avec LOB invariant
                \end{itemize}
            \end{itemize}
        \end{column}
        \begin{column}{0.4\textwidth}
            \begin{alertblock}{Mécanismes de stabilisation}
                \begin{itemize}
                    \item $\theta$: stabilisation locale
                    \item $\theta_{\text{reinit}}$: adaptation globale
                    \item Loi invariante du LOB
                \end{itemize}
            \end{alertblock}
        \end{column}
    \end{columns}
\end{frame}

\begin{frame}{Faits stylisés du modèle}
    \begin{columns}
        \begin{column}{0.5\textwidth}
            \textbf{Ordres limites (Add)}
            \begin{itemize}
                \item \small{Q1: Intensité quasi-constante}
                \begin{itemize}
                    \item \footnotesize{Plus faible à 0 (risque)}
                    \item \footnotesize{Lié au spread et volatilité}
                \end{itemize}
                \item \small{Q2/Q3: Décroissante en taille}
                \begin{itemize}
                    \item \footnotesize{Stratégie de priorité}
                \end{itemize}
            \end{itemize}
            
            \textbf{Annulations (Cancel)}
            \begin{itemize}
                \item \small{Q1: Concave puis plateau}
                \begin{itemize}
                    \item \footnotesize{Valeur de priorité FIFO}
                \end{itemize}
                \item \small{Q2: Asymptote plus rapide}
                \item \small{Q3: Quasi-linéaire}
            \end{itemize}
        \end{column}
        \begin{column}{0.5\textwidth}
            \textbf{Ordres marché (Trade)}
            \begin{itemize}
                \item \small{Q1: Décroissance exponentielle}
                \begin{itemize}
                    \item \footnotesize{Rush sur liquidité rare}
                    \item \footnotesize{Attente si abondante}
                \end{itemize}
                \item \small{Q2: Similaire à Q1}
                \begin{itemize}
                    \item \footnotesize{Uniquement si Q1=0}
                \end{itemize}
                \item \small{Q3: Rare, spread large}
            \end{itemize}
            
            \begin{alertblock}{Implications}
                \begin{itemize}
                    \item Priorité FIFO cruciale
                    \item Comportements stratégiques
                    \item Mean-reversion naturelle
                \end{itemize}
            \end{alertblock}
        \end{column}
    \end{columns}
\end{frame}

\begin{frame}{Probabilité de saut $\theta$}
    \begin{columns}
        \begin{column}{0.5\textwidth}
            \begin{tikzpicture}[scale=0.8]
                % État initial
                \draw[thick] (0,4) rectangle (2,5) node[midway] {$Q_1=100$};
                \draw[thick] (0,3) rectangle (2,4) node[midway] {$Q_{-1}=50$};
                \draw[dashed] (1,2.5) -- (1,5.5);
                \node at (1,5.8) {$p_{ref}$};
                
                % Flèche
                \draw[->, thick] (2.5,4) -- (3.5,4);
                \node[above] at (3,4) {Trade};
                
                % État après trade
                \draw[thick] (4,4) rectangle (6,5) node[midway] {$Q_1=100$};
                \draw[thick] (4,3) rectangle (6,4) node[midway] {$Q_{-1}=0$};
                \draw[dashed] (5,2.5) -- (5,5.5);
                \node at (5,5.8) {$p_{ref}$};
            \end{tikzpicture}
        \end{column}
        \begin{column}{0.5\textwidth}
            \begin{tikzpicture}[scale=0.8]
                % Probabilité theta
                \draw[->] (0,0) -- (2,1) node[right] {$\theta$};
                \draw[->] (0,0) -- (2,-1) node[right] {$1-\theta$};
                
                % États résultants
                \begin{scope}[shift={(3,1)}]
                    \draw[thick] (0,0) rectangle (2,1) node[midway] {$Q_1=100$};
                    \draw[thick] (0,-1) rectangle (2,0) node[midway] {$Q_{-1}=0$};
                    \draw[dashed] (1,-1.5) -- (1,1.5);
                    \node at (1,1.8) {$p_{ref}+\delta$};
                \end{scope}
                
                \begin{scope}[shift={(3,-2)}]
                    \draw[thick] (0,0) rectangle (2,1) node[midway] {$Q_1=100$};
                    \draw[thick] (0,-1) rectangle (2,0) node[midway] {$Q_{-1}=0$};
                    \draw[dashed] (1,-1.5) -- (1,1.5);
                    \node at (1,1.8) {$p_{ref}-\delta$};
                \end{scope}
            \end{tikzpicture}
        \end{column}
    \end{columns}
\end{frame}

\begin{frame}{Exemple de sauts consécutifs}
    \begin{columns}
        \begin{column}{0.6\textwidth}
            \begin{tikzpicture}[scale=0.7]
                % Axe temporel
                \draw[->, thick] (-0.5,0) -- (10,0) node[right] {temps};
                \foreach \x/\t in {0/t_0, 3/t_1, 6/t_2, 9/t_3} {
                    \draw (\x,-0.2) -- (\x,0.2);
                    \node[below] at (\x,-0.2) {$\t$};
                }
                
                % État à t0
                \begin{scope}[shift={(0,1)}]
                    \draw[thick] (0,0) rectangle (1.5,0.8) node[midway] {\small $Q_1=100$};
                    \draw[thick] (0,-0.8) rectangle (1.5,0) node[midway] {\small $Q_{-1}=50$};
                    \draw[dashed] (0.75,-1.2) -- (0.75,1.2);
                    \node[above] at (0.75,1.2) {$p_{ref}$};
                \end{scope}
                
                % État à t1
                \begin{scope}[shift={(3,1)}]
                    \draw[thick] (0,0) rectangle (1.5,0.8) node[midway] {\small $Q_1=100$};
                    \draw[thick] (0,-0.8) rectangle (1.5,0) node[midway] {\small $Q_{-1}=0$};
                    \draw[dashed] (0.75,-1.2) -- (0.75,1.2);
                    \node[above] at (0.75,1.2) {$p_{ref}+\delta$};
                \end{scope}
                
                % État à t2
                \begin{scope}[shift={(6,1)}]
                    \draw[thick] (0,0) rectangle (1.5,0.8) node[midway] {\small $Q_1=0$};
                    \draw[thick] (0,-0.8) rectangle (1.5,0) node[midway] {\small $Q_{-1}=50$};
                    \draw[dashed] (0.75,-1.2) -- (0.75,1.2);
                    \node[above] at (0.75,1.2) {$p_{ref}+2\delta$};
                \end{scope}
                
                % État à t3
                \begin{scope}[shift={(9,1)}]
                    \draw[thick] (0,0) rectangle (1.5,0.8) node[midway] {\small $Q_1=0$};
                    \draw[thick] (0,-0.8) rectangle (1.5,0) node[midway] {\small $Q_{-1}=0$};
                    \draw[dashed] (0.75,-1.2) -- (0.75,1.2);
                    \node[above] at (0.75,1.2) {$p_{ref}+3\delta$};
                \end{scope}
                
                % Flèches entre les états
                \draw[->, thick] (1.7,1) -- (2.8,1);
                \draw[->, thick] (4.7,1) -- (5.8,1);
                \draw[->, thick] (7.7,1) -- (8.8,1);
            \end{tikzpicture}
        \end{column}
        \begin{column}{0.4\textwidth}
            \textbf{Calcul de probabilité:}
            \vspace{0.3cm}
            
            Pour 3 sauts consécutifs vers le haut:
            \begin{align*}
                P &= \theta \cdot \theta \cdot \theta \\
                &= \theta^3
            \end{align*}
            \vspace{0.2cm}
            Avec $\theta \approx 0.7$:
            \begin{align*}
                P &\approx 0.343
            \end{align*}
            \vspace{0.2cm}
            Cette probabilité relativement faible explique l'effet de mean-reversion observé sur les marchés.
        \end{column}
    \end{columns}
\end{frame}

% Section 3: Théorèmes principaux
\section{Théorèmes principaux}

\begin{frame}{Convergence du quantile}
    \begin{theorem}[Convergence du quantile]
        La valeur $\Delta_T^{n_T}(\alpha)$ tend vers une constante $\Delta_T(\alpha)$ qui ne dépend plus de $T_f$:
        \begin{align*}
            T_f &\underset{n\to +\infty}{\to}+\infty \ \text{p.s} \\
            \frac{n_T}{n} &\underset{n\to +\infty}{\to}C^{ste} \ \text{p.s} \\
            \Delta_T^{n_T}(\alpha) &\underset{n\to +\infty}{\to}\Delta_T(\alpha) \ \text{p.s}
        \end{align*}
    \end{theorem}
\end{frame}

\begin{frame}{Uniformité des événements rares}
    \begin{theorem}[Uniformité des événements rares]
        Pour $X_k$ le nombre d'événements rares dans une fenêtre glissante:
        \[X_k \sim \mathcal{B}(N, \alpha)\]
        avec intervalle de Wilson pour la proportion $p_k$:
        \[IC_{\text{Wilson}} = \left[ \frac{\hat{p} + \frac{z^2}{2N} \pm z \sqrt{\frac{\hat{p}(1-\hat{p})}{N} + \frac{z^2}{4N^2}}}{1 + \frac{z^2}{N}} \right]\]
    \end{theorem}
\end{frame}

% Section 4: Analyse empirique
\section{Analyse empirique}

\begin{frame}{Données utilisées}
    \begin{itemize}
        \item Trois actifs étudiés:
        \begin{itemize}
            \item GOOGL: 2.67M actions/jour (49.65\% Add)
            \item LCID: 70.7k actions/jour (45.69\% Add)
            \item KHC: 203k actions/jour (47.75\% Add)
        \end{itemize}
        \item Période: 65 jours (28 juillet - 4 novembre)
        \item Source: Databento (Nasdaq)
    \end{itemize}
\end{frame}

\section{Hypothèses de recherche et résultats empiriques sur GOOGL}

\begin{frame}{Hypothèses théoriques (1/2)}
    \begin{itemize}
        \item \textbf{Impact sur les intensités}:
        \begin{itemize}
            \item Augmentation $\lambda^L$ post-news
            \item Modification $\lambda^C$ (repositionnement)
            \item Variation $\lambda^M$ selon nature de l'information
            \item Asymétrie côté carnet impacté
        \end{itemize}
        \item \textbf{Structure du LOB}:
        \begin{itemize}
            \item Déformation temporaire des queues
            \item Modification ratios $q_1/q_{-1}$
            \item Ajustement queues secondaires
            \item Changement probabilité $p_{\text{ref}}$
        \end{itemize}
    \end{itemize}
\end{frame}

\begin{frame}{Hypothèses théoriques (2/2)}
    \begin{itemize}
        \item \textbf{Comportements stratégiques}:
        \begin{itemize}
            \item Phase 1: Annulations + Ordres marché
            \item Phase 2: Repositionnement progressif
            \item Retour à l'��quilibre des intensités
        \end{itemize}
        \item \textbf{Impact sur la liquidité}:
        \begin{itemize}
            \item Court terme:
            \begin{itemize}
                \item Augmentation spread
                \item Baisse profondeur
                \item Hausse volatilité
            \end{itemize}
            \item Moyen terme:
            \begin{itemize}
                \item Retour à l'équilibre
                \item Meilleure efficience informationnelle
            \end{itemize}
        \end{itemize}
    \end{itemize}
\end{frame}

\begin{frame}{Prix et limites de GOOGL}
    \begin{columns}
        \begin{column}{0.7\textwidth}
            \begin{figure}
                \centering
                \includegraphics[width=0.25\textwidth]{Prix&limit.png}
                \caption{Évolution du prix de GOOGL avec le bid-ask (2024-07-25, 16:02-16:03). Les points noirs sont les trades, les tracés sont les limites -9 à +9}
            \end{figure}
        \end{column}
        \begin{column}{0.3\textwidth}
            \begin{alertblock}{Validation du modèle QR}
                \begin{itemize}
                    \item Mouvements de prix conformes aux prédictions
                    \item Mean-reversion observée
                    \item Sauts de prix cohérents avec $\theta$
                    \item Imbalances prédictives
                \end{itemize}
            \end{alertblock}
        \end{column}
    \end{columns}
\end{frame}

\begin{frame}{Intensités des événements}
    \begin{figure}
        \centering
        \includegraphics[width=0.25\textwidth]{trades_imbalnces_intensity.png}
        \caption{Intensité des Trades en fonction de l'imbalance pour GOOGL (IC 95\%, première limite)}
    \end{figure}
    \begin{itemize}
        \item Symétrie de l'imbalance entre ask et bid
        \item Intensité plus élevée aux imbalances extrêmes
        \item Les acteurs exploitent le MBO aux imbalances extrêmes
    \end{itemize}
\end{frame}

\begin{frame}{Add et Cancel en fonction de l'imbalance}
    \begin{figure}
        \centering
        \includegraphics[width=0.25\textwidth]{ADD_CANCEL_INTENSITY_IMBALANCE_GOOGL.png}
        \caption{Intensité des Add (bleu) et Cancel (vert) vs imbalance pour GOOGL (IC 95\%, seconde limite)}
    \end{figure}
\end{frame}

\begin{frame}{Événements majeurs (1/3)}
    \begin{columns}
        \begin{column}{0.48\textwidth}
            \begin{figure}
                \centering
                \includegraphics[width=\textwidth]{Graphe_1.png}
                \caption{20 sept. 2024: Plainte UK monopole\\
                Impact fort et immédiat sur le LOB}
            \end{figure}
        \end{column}
        \begin{column}{0.48\textwidth}
            \begin{figure}
                \centering
                \includegraphics[width=\textwidth]{Graphe4.png}
                \caption{25 juil. 2024: Résultats Q2 + ChatGPT\\
                Double impact: financier et technologique}
            \end{figure}
        \end{column}
    \end{columns}
\end{frame}

\begin{frame}{Événements majeurs (2/3)}
    \begin{columns}
        \begin{column}{0.48\textwidth}
            \begin{figure}
                \centering
                \includegraphics[width=\textwidth]{Graphe5.png}
                \caption{17 nov. 2024: Gemini + Pixel 9\\
                Impact technologique majeur}
            \end{figure}
        \end{column}
        \begin{column}{0.48\textwidth}
            \begin{figure}
                \centering
                \includegraphics[width=\textwidth]{Graphe6.png}
                \caption{29 août 2024: Faille Chrome\\
                Réaction rapide mais limitée}
            \end{figure}
        \end{column}
    \end{columns}
\end{frame}

\begin{frame}{Événements majeurs (3/3)}
    \begin{columns}
        \begin{column}{0.48\textwidth}
            \begin{figure}
                \centering
                \includegraphics[width=\textwidth]{Graphe7.png}
                \caption{18 sept. 2024: Cut FED\\
                Impact macroéconomique global}
            \end{figure}
        \end{column}
        \begin{column}{0.48\textwidth}
            \begin{figure}
                \centering
                \includegraphics[width=\textwidth]{Grpahe7.png}
                \caption{5 août 2024: Amende US\\
                Impact réglementaire direct}
            \end{figure}
        \end{column}
    \end{columns}
\end{frame}

\begin{frame}{Synthèse des événements majeurs}
    \begin{columns}
        \begin{column}{0.48\textwidth}
            \begin{alertblock}{Caractéristiques communes}
                \begin{itemize}
                    \item Asymétrie des réactions selon nature de la news
                    \item Temps de réaction différents selon l'événement
                    \item Impact sur la profondeur du carnet
                    \item Réaction instantanée du marché
                    \item Modification profonde de la structure du LOB
                    \item Persistance des effets sur plusieurs minutes
                \end{itemize}
            \end{alertblock}
        \end{column}
        \begin{column}{0.48\textwidth}
            \begin{alertblock}{Synthèse des impacts}
                \begin{itemize}
                    \item Hiérarchie claire des impacts selon type de news
                    \item Patterns de réaction spécifiques
                    \item Corrélation avec volume d'échanges
                    \item Classification possible des événements:
                    \begin{itemize}
                        \item Impacts réglementaires
                        \item Impacts technologiques
                        \item Impacts macroéconomiques
                    \end{itemize}
                \end{itemize}
            \end{alertblock}
        \end{column}
    \end{columns}
\end{frame}

\begin{frame}{Microstructure des perturbations}
    \begin{columns}
        \begin{column}{0.6\textwidth}
            \textbf{Changement de régime lors d'une news blbla}
            \begin{itemize}
                \item \textbf{Phase pré-perturbation blbla}:
                \begin{itemize}
                    \item Structure normale du LOB blbla
                    \item Intensités standards blbla
                \end{itemize}
                \item \textbf{Phase de transition blbla}:
                \begin{itemize}
                    \item Rupture brutale blbla
                    \item Modification des intensités blbla
                    \item Réorganisation du carnet blbla
                \end{itemize}
            \end{itemize}
        \end{column}
        \begin{column}{0.4\textwidth}
            \begin{alertblock}{Visualisation}
                \centering
                \textit{IMAGE A METTRE DE GOOGL\\
                PILE AU MOMENT DU\\
                CHANGEMENT DE REGIME\\
                DE NEWS}
            \end{alertblock}
        \end{column}
    \end{columns}
    \vspace{0.2cm}
    \begin{itemize}
        \item \textbf{Caractéristiques observées blbla}:
        \begin{itemize}
            \item Discontinuité des intensités blbla
            \item Modification du ratio bid/ask blbla
            \item Nouvelle structure d'équilibre blbla
        \end{itemize}
    \end{itemize}
\end{frame}

% Remerciements
\begin{frame}{Remerciements}
    \begin{center}
        \Large Merci de votre attention
        \vspace{1cm}
        
        \normalsize
        Questions?
    \end{center}
\end{frame}

\end{document}