% Classe du document
\documentclass[aspectratio=169]{beamer}  % Format 16:9 pour un affichage moderne

% Thème et couleurs
\usetheme{Madrid}
\usecolortheme{whale}
\useinnertheme{rectangles}
\useoutertheme{miniframes}
\setbeamertemplate{navigation symbols}{}  % Supprime les symboles de navigation

% Packages essentiels
\usepackage[utf8]{inputenc}
\usepackage[T1]{fontenc}
\usepackage[french]{babel}
\usepackage{lmodern}
\usepackage{amsmath,amssymb,amsthm}
\usepackage{graphicx}
\usepackage{pgfplots}
\usepackage{listings}  % Pour le code
\pgfplotsset{compat=1.18}

% Informations du document
\title[Impact des nouvelles sur les LOB]{Impact des nouvelles sur les \textit{Limit Order Books}}
\subtitle{Une analyse par le modèle Queue-Reactive}
\author[LAFERTE \& AIAD]{LAFERTE Edouard \and AIAD Janis}
\institute[École Polytechnique]{
    Département de mathématiques appliquées\\
    École Polytechnique
}
\date{Juin 2024}

\begin{document}

% Page de titre
\begin{frame}
    \titlepage
\end{frame}

% Plan
\begin{frame}{Plan de la présentation}
    \tableofcontents
\end{frame}

% Section 0: Données et Méthodologie
\section{Données et Méthodologie}

\begin{frame}{Sources des données}
    \begin{itemize}
        \item \textbf{Nasdaq Databento}:
        \begin{itemize}
            \item Données haute fréquence (microsecondes)
            \item Format parquet optimisé
            \item Messages d'ordres complets (ITCH)
        \end{itemize}
        \item \textbf{Chicago Mercantile Exchange}:
        \begin{itemize}
            \item Données futures et options
            \item Granularité temporelle fine
        \end{itemize}
    \end{itemize}
\end{frame}

\begin{frame}{Prétraitement des données}
    \begin{itemize}
        \item \textbf{Pipeline de nettoyage}:
        \begin{itemize}
            \item Filtrage des données aberrantes
            \item Synchronisation temporelle
            \item Reconstruction du carnet d'ordres
        \end{itemize}
        \item \textbf{Infrastructure}:
        \begin{itemize}
            \item Cluster SLURM pour calculs parallèles
            \item Optimisation GPU (CUDA)
            \item Décomposition de Cholesky pour matrices
        \end{itemize}
    \end{itemize}
\end{frame}

\begin{frame}{Infrastructure technique}
    \begin{itemize}
        \item \textbf{Environnement de calcul}:
        \begin{itemize}
            \item Jupyter notebooks interactifs
            \item Calculs distribués sur cluster
            \item Optimisation mémoire et CPU
        \end{itemize}
        \item \textbf{Librairie open-source}:
        \begin{itemize}
            \item github.com/janisaiad/HFT\_QR\_RL
            \item Implémentation du modèle QR
            \item Outils d'analyse et visualisation
        \end{itemize}
    \end{itemize}
\end{frame}

\begin{frame}{Caractéristiques des données}
    \begin{table}
    \centering
    \begin{tabular}{|c|c|c|c|c|}
    \hline
    \textbf{Actif} & \textbf{Nb/Jour Actions} & \textbf{Pct Add} & \textbf{Pct Cancel} & \textbf{Pct Order} \\ \hline
    GOOGL & 2.67M & 49.65\% & 46.64\% & 3.70\% \\ \hline
    LCID & 70.7k & 45.69\% & 42.94\% & 11.35\% \\ \hline
    KHC & 203k & 47.75\% & 45.34\% & 6.89\% \\ \hline
    \end{tabular}
    \caption{Statistiques moyennes sur 65 jours (28 juillet - 4 novembre)}
    \end{table}
    \vspace{0.3cm}
    \begin{itemize}
        \item Données haute fréquence (microsecondes)
        \item Format parquet optimisé
        \item Messages d'ordres complets (ITCH)
        \item Période d'étude : 65 jours
    \end{itemize}
\end{frame}

\begin{frame}{Critères de filtrage des données}
    \begin{table}
    \centering
    \begin{tabular}{|c|c|c|}
    \hline
    \textbf{Critère} & \textbf{Seuil} & \textbf{Impact} \\ \hline
    Outliers & $\pm3\sigma$ des prix & Suppression points aberrants \\ \hline
    Missing data & $>5\%$ par jour & Interpolation linéaire \\ \hline
    Valeurs aberrantes & 614 dans les tailles & Remplacement par 0 \\ \hline
    Profondeur carnet & Niveau 0 & Analyse du BBO \\ \hline
    Publisher ID & 39 & Sélection publisher\_id 39 \\ \hline
    Horodatage & ts\_event & 14h-20h uniquement \\ \hline
    \end{tabular}
    \end{table}
    \vspace{0.3cm}
    \begin{itemize}
        \item Focus sur un titre unique (LCID)
        \item Validation systématique des modifications
        \item Contrôle de cohérence des actions
    \end{itemize}
\end{frame}

\begin{frame}{Pipeline de nettoyage}
    \begin{columns}
        \begin{column}{0.48\textwidth}
            \textbf{Étapes de validation}:
            \begin{itemize}
                \item Transactions (T):
                \begin{itemize}
                    \item Vérification volume échangé
                    \item Cohérence des diminutions
                \end{itemize}
                \item Ajouts (A):
                \begin{itemize}
                    \item Validation taille ajoutée
                    \item Correspondance exacte
                \end{itemize}
            \end{itemize}
        \end{column}
        \begin{column}{0.48\textwidth}
            \textbf{Traitement technique}:
            \begin{itemize}
                \item Cast en Int64
                \item Vérification colonnes
                \item Indicateurs qualité:
                \begin{itemize}
                    \item status\_N
                    \item status\_diff
                \end{itemize}
            \end{itemize}
        \end{column}
    \end{columns}
\end{frame}

% Section 1: Introduction aux Limit Order Books
\section{Introduction aux Limit Order Books}

\begin{frame}{Les marchés haute fréquence}
    \begin{itemize}
        \item Électronification des marchés financiers
        \item Réduction drastique du temps d'exécution
        \item Trois catégories de trading:
        \begin{itemize}
            \item \textbf{Low Frequency (LF)}: Échelle de plusieurs mois
            \item \textbf{Mid Frequency (MF)}: Échelle journalière
            \item \textbf{High Frequency (HF)}: Échelle microseconde
        \end{itemize}
    \end{itemize}
\end{frame}

\begin{frame}{Structure du Limit Order Book}
    \begin{columns}
        \begin{column}{0.6\textwidth}
            \begin{itemize}
                \item Trois types d'ordres:
                \begin{itemize}
                    \item \textbf{Limit Order}: Proposition à prix fixé
                    \item \textbf{Market Order}: Exécution immédiate
                    \item \textbf{Cancellation}: Retrait d'ordre
                \end{itemize}
                \item Organisation en files d'attente
                \item Prix de référence $p_{ref}$
            \end{itemize}
        \end{column}
        \begin{column}{0.4\textwidth}
            % Insérer ici un schéma simplifié du LOB
        \end{column}
    \end{columns}
\end{frame}

% Section 2: Le modèle Queue-Reactive
\section{Le modèle Queue-Reactive}

\begin{frame}{Principes du modèle QR}
    \begin{itemize}
        \item Modélisation par files d'attente
        \item Prix fixe de référence $p_{ref}$
        \item Évolution selon processus de Markov
        \item Variables de Poisson indépendantes
    \end{itemize}
\end{frame}

\begin{frame}{Dynamique du modèle}
    \begin{equation*}
        Q(t) = (Q_{-K}(t),...Q_{-1}(t),Q_{1}(t),Q_K(t))
    \end{equation*}
    \begin{itemize}
        \item Intensités $\lambda_i^U(Q_i(t))$
        \item Choix de l'action: minimum des variables de Poisson
        \item Paramètre de mean-reversion $\theta$
    \end{itemize}
\end{frame}

% Section 3: Théorèmes principaux
\section{Théorèmes principaux}

\begin{frame}{Convergence du quantile}
    \begin{theorem}[Convergence du quantile]
        La valeur $\Delta_T^{n_T}(\alpha)$ tend vers une constante $\Delta_T(\alpha)$ qui ne dépend plus de $T_f$:
        \begin{align*}
            T_f &\underset{n\to +\infty}{\to}+\infty \ \text{p.s} \\
            \frac{n_T}{n} &\underset{n\to +\infty}{\to}C^{ste} \ \text{p.s} \\
            \Delta_T^{n_T}(\alpha) &\underset{n\to +\infty}{\to}\Delta_T(\alpha) \ \text{p.s}
        \end{align*}
    \end{theorem}
\end{frame}

\begin{frame}{Uniformité des événements rares}
    \begin{theorem}[Uniformité des événements rares]
        Pour $X_k$ le nombre d'événements rares dans une fenêtre glissante:
        \[X_k \sim \mathcal{B}(N, \alpha)\]
        avec intervalle de Wilson pour la proportion $p_k$:
        \[IC_{\text{Wilson}} = \left[ \frac{\hat{p} + \frac{z^2}{2N} \pm z \sqrt{\frac{\hat{p}(1-\hat{p})}{N} + \frac{z^2}{4N^2}}}{1 + \frac{z^2}{N}} \right]\]
    \end{theorem}
\end{frame}

% Section 4: Analyse empirique
\section{Analyse empirique}

\begin{frame}{Données utilisées}
    \begin{itemize}
        \item Trois actifs étudiés:
        \begin{itemize}
            \item GOOGL: 2.67M actions/jour (49.65\% Add)
            \item LCID: 70.7k actions/jour (45.69\% Add)
            \item KHC: 203k actions/jour (47.75\% Add)
        \end{itemize}
        \item Période: 65 jours (28 juillet - 4 novembre)
        \item Source: Databento (Nasdaq)
    \end{itemize}
\end{frame}

\begin{frame}{L'imbalance: mesure clé}
    \begin{equation*}
        \text{Imb}_t = \frac{Q^{best\ bid}_t-Q^{best\ ask}_t}{Q^{best\ bid}_t+Q^{best\ ask}_t}
    \end{equation*}
    \begin{itemize}
        \item Indicateur de déséquilibre
        \item Prédicteur de mouvements de prix
        \item Distribution non uniforme
    \end{itemize}
\end{frame}

% Section 5: Conclusions
\section{Conclusions}

\begin{frame}{Conclusions principales}
    \begin{itemize}
        \item Contributions méthodologiques:
        \begin{itemize}
            \item Cadre probabiliste rigoureux
            \item Détection fine des anomalies
        \end{itemize}
        \item Limites du modèle QR standard
        \item Nécessité d'intégrer:
        \begin{itemize}
            \item Composante endogène (LOB)
            \item Composante exogène (news)
        \end{itemize}
    \end{itemize}
\end{frame}

% Remerciements
\begin{frame}{Remerciements}
    \begin{center}
        \Large Merci de votre attention
        \vspace{1cm}
        
        \normalsize
        Questions?
    \end{center}
\end{frame}

\end{document}